\documentclass[english,12pt,a4paper,pdftex]{article}

%% Nämä komennot asettavat oikean tekstiencodauksen.
\usepackage[utf8]{inputenc}
\usepackage[OT1]{fontenc}

\usepackage{natbib}

\bibpunct{(}{)}{;}{a}{,}{,}

\usepackage{verbatim}

%% Tämä paketti on pakollinen
%% Valitse korkeakoulusi näistä: arts, biz, chem, elec, eng, sci.
%%
%% This package is required
%% Choose your school from arts, biz, chem, elec, eng, sci.
\usepackage[sci]{aaltothesis}

%% Jos käytät latex-komentoa käännettäessä (oletusarvo) 
%% kuvat kannattaa tehdä eps-muotoon. Älä käytä ps-muotoisia kuvia!
%% Käytä seuraavaa latex-komennon ja eps-kuvien kanssa 
%%
%% Jos tääs käytät pdflatex-komentoa, joka kääntää tekstin suoraan
%% pdf-tiedostoksi, kuvasi on oltava jpg-formaatissa tai pdf-formaatissa.
%%
%% Use this if you run pdflatex and use jpg/pdf-format pictures.
%%
\usepackage{graphicx}

%% Saat pdf-tiedoston viittaukset ja linkit kuntoon seuraavalla paketilla.
%% Paketti toimii erityisen hyvin pdflatexin kanssa. 
%%
%% Use this if you want to get links and nice output with pdflatex
\usepackage[pdfpagemode=None,colorlinks=false,urlcolor=red,linkcolor=blue,citecolor=black,pdfstartview=FitH]{hyperref}


%% Jos et jostain syystä tykkää käyttää
%% edellistä hyperref pakettia, voit käyttää myös seuraavaa pakettia
%% (tarvitaan lähinnä url-komennon määrittämiseen ja formatoimiseen)
%%
%% Use this if you do not like hyperref package - this
%% defines url environment and formats it correctly
\usepackage{url}

%% Matematiikan fontteja, symboleja ja muotoiluja lisää, näitä tarvitaan usein 
%%
%% Use this if you write hard core mathematics, these are usually needed
% \usepackage{amsfonts,amssymb,amsbsy}  

%% Vaakasuunnan mitat, ÄLÄ KOSKE!
\setlength{\hoffset}{-1in}
\setlength{\oddsidemargin}{35mm}
\setlength{\evensidemargin}{25mm}
\setlength{\textwidth}{15cm}
%% Pystysuunnan mitat, ÄLÄ KOSKE!
\setlength{\voffset}{-1in}
\setlength{\headsep}{7mm}
\setlength{\headheight}{1em}
\setlength{\topmargin}{25mm-\headheight-\headsep}
\setlength{\textheight}{23cm}


%% Kaikki mikä paperille tulostuu, on tämän jälkeen
%%
%% Output starts here
\begin{document}

%% Korjaa vastaamaan korkeakouluasi, jos automaattisesti asetettu nimi on 
%% virheellinen 
%%
%% Change the school field to describe your school if the autimatically 
%% set name is wrong
% \university{aalto University}{aalto-Yliopisto}
% \school{School of Electrical Engineering}{SähköTekniikan korkeakoulu}

%% Vain kandityölle: Korjaa seuraavat vastaamaan koulutusohjelmaasi
%%
%% Only for B.Sc. thesis: Choose your degree programme. 
% \degreeprogram{Electronics and electrical engineering}% {Elektroniikka ja sähkötekniikka}
%%

%% Vain DI/M.Sc.- ja lisensiaatintyölle: valitse laitos, 
%% professuuri ja sen professuurikoodi. 
%%
%% Only for M.Sc. and Licentiate thesis: Choose your department,
%% professorship and professorship code. 
\department{Department of Media Technology}
{Mediatekniikan laitos}
\professorship{Media Technology}{Mediatekniikka}
\code{T-111}
%%

%% Valitse yksi näistä kolmesta
%%
%% Choose one of these:
%\univdegree{BSc}
\univdegree{MSc}
%\univdegree{Lic}

%% Oma nimi
%%
%% Should be self explanatory...
\author{Mikko Koski}

%% Opinnäytteen otsikko tulee vain tähän. Älä tavuta otsikkoa ja
%% vältä liian pitkää otsikkotekstiä. Jos latex ryhmittelee otsikon
%% huonosti, voit joutua pakottamaan rivinvaihdon \\ kontrollimerkillä.
%% Muista että otsikkoja ei tavuteta! 
%% Jos otsikossa on ja-sana, se ei jää rivin viimeiseksi sanaksi 
%% vaan aloittaa uuden rivin.
%% 
%% Your thesis title. If the title is very long and the latex 
%% does unsatisfactory job of breaking the lines, you will have to
%% break the lines yourself with \\ control character. 
%% Do not hyphenate titles.
\thesistitle{Effective communication media for customer feedback in agile software projects}{Tehokkaat kommunikaatiokanavat asiakaspalautteen antamiseen ketterissä ohjelmistoprojekteissa}

\place{Espoo}
%% Kandidaatintyön päivämäärä on sen esityspäivämäärä! 
%% 
%% For B.Sc. thesis use the date when you present your thesis. 
\date{20.3.2012}

%% Kandidaattiseminaarin vastuuopettaja tai diplomityön valvoja.
%% Huomaa tittelissä "\" -merkki pisteen jälkeen, 
%% ennen välilyöntiä ja seuraavaa merkkijonoa. 
%% Näin tehdään, koska kyseessä ei ole lauseen loppu, jonka jälkeen tulee 
%% hieman pidempi väli vaan halutaan tavallinen väli.
%%
%% B.Sc. or M.Sc. thesis supervisor 
%% Note the "\" after the comma. This forces the following space to be 
%% a normal interword space, not the space that starts a new sentence. 
\supervisor{Prof.\ Tapio Takala}{Prof.\ Tapio Takala}

%% Kandidaatintyön ohjaaja(t) tai diplomityön ohjaaja(t)
%% 
%% B.Sc. or M.Sc. thesis advisors(s). 
%%
%% Note that there has been a change in the official EN translation
%% of the Finnish title ``ohjaaja'' which in the previous version (1.5) 
%% of this document was called ``instructor''. The recommended
%% translation is now ``advisor''.  
%% However, the LaTeX internal variable remains \instructor
%% as there is little point to change the variable name. 
%%
%\instructor{Prof. Pirjo Professori}{Prof. Pirjo Professori}
\instructor{Risto Sarvas}{Risto Sarvas}
%\instructor{M.Sc.\ (Tech.) Polli Pohjaaja}{DI Polli Pohjaaja}

%% Aaltologo: syntaksi:
%% \uselogo{aaltoRed|aaltoBlue|aaltoYellow|aaltoGray|aaltoGrayScale}{?|!|''}
%% Logon kieli on sama kuin dokumentin kieli
%%
%% Aalto logo: syntax:
% \uselogo{aaltoRed|aaltoBlue|aaltoYellow|aaltoGray|aaltoGrayScale}{?|!|''}
%% Logo language is set to be the same as the document language.
\uselogo{aaltoRed}{''}

%% Tehdään kansilehti
%%
%% Create the coverpage
\makecoverpage


%% Suomenkielinen tiivistelmä
%% 
%% Finnish abstract
%%
%% Tiivistelmän avainsanat
\keywords{kommunikaatio, ketterä ohjelmistotuotanto, asiakaspalaute}
%% Tiivistelmän tekstiosa
\begin{abstractpage}[finnish]
  Tiivistelmässä on lyhyt selvitys (noin 100 sanaa)
  kirjoituksen tärkeimmästä sisällöstä: mitä ja miten on tutkittu,
  sekä mitä tuloksia on saatu. 
  Tiivistelmässä on lyhyt selvitys (noin 100 sanaa)
  kirjoituksen tärkeimmästä sisällöstä: mitä ja miten on tutkittu,
  sekä mitä tuloksia on saatu. 

  Tiivistelmässä on lyhyt selvitys (noin 100 sanaa)
  kirjoituksen tärkeimmästä sisällöstä: mitä ja miten on tutkittu,
  sekä mitä tuloksia on saatu. 
  Tiivistelmässä on lyhyt selvitys (noin 100 sanaa)
  kirjoituksen tärkeimmästä sisällöstä: mitä ja miten on tutkittu,
  sekä mitä tuloksia on saatu. 
  Tiivistelmässä on lyhyt selvitys (noin 100 sanaa)
  kirjoituksen tärkeimmästä sisällöstä: mitä ja miten on tutkittu,
  sekä mitä tuloksia on saatu. 
\end{abstractpage}

%% Pakotetaan uusi sivu varmuuden vuoksi, jotta 
%% mahdollinen suomenkielinen ja englanninkielinen tiivistelmä
%% eivät tule vahingossakaan samalle sivulle
%%
%% Force new page so that English abstract starts from a new page
\newpage
%
%% English abstract, uncomment if you need one. 
%% 
%% Abstract keywords
\keywords{communication, agile, software, feedback, media}
%% Abstract text
\begin{abstractpage}[english]
 Your abstract in English. Try to keep the abstract short, approximately 
 100 words should be enough. Abstract explains your research topic, 
 the methods you have used, and the results you obtained.  
\end{abstractpage}
%% Note that 
%% if you are writting your master's thesis in English place the English
%% abstract first followed by the possible Finnish abstract

%% Esipuhe 
%%
%% Preface
\mysection{Preface}
TODO: Haluan kiittää Professori Pirjo 
Professoria ja ohjaajaani Olli Ohjaajaa hyvästä ja 
huonosta ohjauksesta.\\

\vspace{5cm}
Otaniemi, 9.3.2012

\vspace{5mm}
{\hfill Mikko Koski \hspace{1cm}}

%% Pakotetaan varmuuden vuoksi esipuheen jälkeinen osa
%% alkamaan uudelta sivulta
%%
%% Force new page after preface
\newpage


%% Sisällysluettelo
%% addcontentsline tekee pdf-tiedostoon viitteen sisällysluetteloa varten
%% 
%% Table of contents. 
%\addcontentsline{toc}{section}{Sisällysluettelo}
\addcontentsline{toc}{section}{Contents}
%% Tehdään sisällysluettelo
%%
%% Create it. 
\tableofcontents


%% Symbolit ja lyhenteet
%%
%% Symbols and abbreviations
% \mysection{Symbolit ja lyhenteet}
% \mysection{Symbols and abbreviations}

% \subsection*{Lyhenteet}

\clearpage

\subsection*{Abbreviations}

\begin{tabular}{ll}
MRT         & Media Richness Theory \\
MST         & Media Synchronocity Theory \\
API         & Application Programming Interface
\end{tabular}


%% Sivulaskurin viilausta opinnäytteen vaatimusten mukaan:
%% Aloitetaan sivunumerointi arabialaisilla numeroilla (ja jätetään
%% leipätekstin ensimmäinen sivu tyhjäksi, 
%% ks. alla \thispagestyle{empty}).
%% Pakotetaan lisäksi ensimmäinen varsinainen tekstisivu alkamaan 
%% uudelta sivulta clearpage-komennolla. 
%% clearpage on melkein samanlainen kuin newpage, mutta 
%% flushaa myös LaTeX:n floatit 
%% 
%% Corrects the page numbering, there is no need to change these
\cleardoublepage

\storeinipagenumber
\pagenumbering{arabic}
\setcounter{page}{1}

\clearpage

%% Leipäteksti alkaa
%%
%% Text body begins. Note that since the text body
%% is mostly in Finnish the majority of comments are
%% also in Finnish after this point. There is no point in explaining
%% Finnish-language specific thesis conventions in English.
\section{Introduction}

Agile software development in its essence is all about feedback. The core principle of agile development is to have short iterations and deliver a possibly shippable product increment after each iteration \citep{schwaber2009agile}. The delivered product increment enables a possibility for customer to give feedback about the outcome of the iteration and this way direct the development organization to the correct route.

Since the rise of agile software development methods, customer communication and customer collaboration have been taken seriosly and they have been identified as one of the key elements in successful software projects \citep{agilemanifesto}. Also, in previous research it has been shown that a lack of communication and customer involvement is one of the biggest challenges faces by agile teams \citep{korkala2006}.

A lot of research has been conducted about communication is software projects but not so many with the focus on some specific aspect of communication. I believe that customer communication in software projects is a wide subject that includes different types of communication methods in differect situations. For example, the communication required while doing planning is very different from the communication required while customer is givin feedback. Thus, it makes sense to focus on specific area of communication.

Tämä luku vastaa kysymykseen:

\begin{itemize}
\item What's the research problem, bigger phenomenon
\item Why does the problem matter?
\end{itemize}

%% Ensimmäinen sivu tyhjäksi
%% 
%% Leave first page empty
\thispagestyle{empty}



%% Opinnäytteessä jokainen osa alkaa uudelta sivulta, joten \clearpage
%%
%% In a thesis, every section starts a new page, hence \clearpage

\clearpage

\section{Literature}

Vastataan kysymykseen:

\begin{itemize}
\item What has been studied before?
\item What hasn't been studied yet?
\end{itemize}

\subsection{Definitions}

In this section the most revelant terms and concepts for the thesis are introduces and defined.

\subsubsection{Customer feedback}

\begin{comment}
\begin{itemize}
\item What does \textit{customer} mean?
\item What does \textit{feedback} mean? Feedback from what?
\end{itemize}
\end{comment}

In this thesis \textit{customer} refers to a person in a software project who is the feedback provider. In the context of external software projects customer refers to a representative from the customer organization who gives feedback to the supplier organization. In the context of internal software project customer refers to the representative in the internal organization who is in response of the project outcode and who provides the feedback to the development team.

As the context of the thesis is in agile software development, by customer I refer to Product Owner. Product Owner is a agile team member who is responsible of product outcome. Product Owner maintains the product backlog, scope and prioritizes the items in the backlog. Because Product Owner is the one who is responsible of the project outcome, she is also a person who most likely provides the team with the most valuable feedback \citep{pichler2010}.

\textit{Feedback} is a part of the communication between customer and the software supplier. It is a phase of customer-supplier communication that can happen only after the supplier has delivered something concreate to the customer. Obviously, if supplier has not delivered yet anything, there is very little for customer to give feedback from. Thus, it can be argued, that feedback is a form of communication that happens only after the project has been going on for some time.

Feedback can be given from various subjects in software projects. Feedback can be given for example from the working practices, working processes, design documents, user-interface drafts or working piece of software. In this thesis the main focus is in feedback given about the working piece of software which has been delivered to the customer. There are various ways how software can be delivered from development team to customer (e.g. DVDs, email etc.). However, in agile software projects the preferred way to deliver software to customer to test it is a Continuous Integration or a staging server. \citep{CI lähde} Thus, in this thesis it is assumed that the feedback is given from a software that is running or otherwise available from a testing server which is updated real-time as the development goes on.

\subsubsection{Effective communication}

\begin{itemize}
\item What does \textit{effective} mean? 
\item What makes communication efective? 
\item What are the properties (in this paper) that make communication efective
\end{itemize}

Kirjoita tähän, että sosiaaliset hyödyt eivät kuulu tähän kategoriaan. Työssä mitataan vain informaation siirtymistä ja sen tehokkuutta.

\subsubsection{Agile software projects}

\begin{itemize}
\item What does \textit{agile} mean in this paper?
\end{itemize}

\subsection{Communication in agile software projects}

\begin{itemize}
\item What has been studied about communication?
\item What has been studied about agile software projects?
\item Why hasn't feedback been studied?
\end{itemize}

\subsection{Media theories}

\textbf{Media Fitness Theory} is a rather new and interesting theory. It was developed by \citet{higa2007}. The theory was selected as a port of this thesis because it mainly focuses to the media selection. Media selection is especially important from the point of this thesis. Understanding the reasons why one medium is preferred over another helps to understand what are the properties of a medium that makes it better fit for a communication task in question.

The theory of Media Fitness is another theory that tries to address to the mismatch between the previously formed theories and the empirical evidence of media selection. The theory is influenced by Media Richness by \citet{daft1986} Theory and Social Influence Perspectives by Fulk et al. \textbf{LÄHDE}.

The theory has been empirically proven to provide rather good match between the theorethical prediction of media selection and the actual choice \citep{higa2007} \citep{gu2011}.

Tässä luvussa esitellään teoriat, joiden pohjalta arvioidaan kommunikaatiotyökaluja.

Etsitään vastauksia mm. seuraaviin kysymyksiin:

Perustele miksi "social influence" on tai ei ole otettu huomioon.

Muita hylättyjä: 
- Relatioship development, Walther 1992 (Interpersonal Effects in Computer-Mediated Interaction A Relational Perspective)
- Channel expansion, Carlson and Zmud 1999 (Channel expansion theory and the experiental nature of media richness perceptions)
- Zigur buckland, task-technology fit
- Nunamaker, Dennis, Valacich - Electronic meeting systems to support group work

\begin{itemize}
\item Millaisia ominaisuuksia hyvässä kommunikaatiotyökalussa on?
\item Miten näitä ominaisuuksia pystyttäisiin tuomaan mukaan tietokoneavusteiseen kommunikaatiotyökaluun (vai pystyykö)
\end{itemize}

\subsubsection{Media Richness Theory}

\begin{comment}
\begin{itemize}
\item What's the theory all about?
\item What does the theory say about communication media
\end{itemize}
\end{comment}

The theory of media richness was proposed by \citet{daft1986}. The theory is well-known and widely supported. However, it has been faces a lot of critisism \citep{dennis1999} \citep{korkala2006}. New theories, such as media synchronicity theory and media naturalness theory have arised to answer the critisism media richness theory has faced.

The main assertion of media richness theory is that different communication methods can be classified either high or low in their "richness" based on their capacity to facilitate shared meaning. In order of decreasing richness, the media classifications are face-to-face, telephone, written personal documents such as letters or memos, impersonal and unaddressed written documents such as fliers or standard formal reports. The hierarchycal classification illustrated figure~\ref{fig:hierarchy_of_media_richness}.

\begin{figure}[htb]
\begin{center}
\includegraphics[width=1.0\textwidth]{hierarchy_of_media_richness.png}
\end{center}
\caption{Hierarchy of media richness \citep{daft1987}}
\label{fig:hierarchy_of_media_richness}
\end{figure}

The theory also utilizes a concept of message uncertainty and equivocality. \textbf{Uncertainty} exists if information can be interpreted unambiguously but there is a lack of information. In other words, uncertainty has come to mean \textit{absence of information}. Uncertainty has also been defined as the difference between the amount of information required to perform the task and the amount of information already possessed by the organization. Uncertainty can be reduced by acquiring more information to support the decision making. Managers in organizations can for example simply ask questions to gain more knowledge and thus reduce the uncertainty \citep{daft1987}. In contract, \textbf{equivocality} means \textit{ambiguity}, the existence of multiple and conflicting interpretations, even though the amount of information available is sufficient \citep{daft1987}. Equivocality means confusion and lack of understanding and it can not be reduced by acquiring more information. Gathering more information may be even impossible since the managers may not be certain what questions to ask. The higher the level of equivocality is, the more negotiation is required to reach a consensus on one interpretation.

The media richness theory lists four criteria which define the richness of a communication media. The criteria are feedback, multiple cues, language variety and personal focus. Even though the media richness theory does not include new online media \citet{graveline2000} have extended the theory and the four criteria to include the new online media. According to \citet{graveline2000} the four criteria are described in the table \ref{table:criteria_media_richness}.

\begin{table}[!h]
% increase table row spacing, adjust to taste
\renewcommand{\arraystretch}{1.3}
% if using array.sty, it might be a good idea to tweak the value of
% \extrarowheight as needed to properly center the text within the cells
\caption{Four criteria to define the media richness \citep{graveline2000} \citep{daft1987}}
\label{table:criteria_media_richness}
\centering
% Some packages, such as MDW tools, offer better commands for making tables
% than the plain LaTeX2e tabular which is used here.
\begin{tabular}{|p{4cm}|p{10cm}|}
\hline
\textbf{Media character} & \textbf{Description}\\
\hline
Feedback capability & How quickly communication participants can react to the transmitted message e.g. by asking questions and making corrections. The capability of feedback relates to synchronicity of feedback. Face-to-face communication has high feedback capability which exchanging documents has low feedback capability. Online media can be either synchronous or asynchronous. Synchronous media, e.g. video conference have high feedback capability where e.g. bulletin board or email have low feedback capability. \\
\hline
Availability of multiple cues & The richness of various communication channels available to the participants i.e. physical presence, body language and voice inflection. Some online media are capable of transmitting multiple cues (e.g. videoconference) while some are primarily single-channel (email, text chat) \\
\hline
Language variety & The range of meaning that can be conveyed with language symbols. Numbers convey greater precision of meaning than does natural language. Natural language can be used to convey understanding of a broader set of concepts and ideas \\
\hline
Personal focus & Level of individual attention and personal feelings the message contains \\
\hline
\end{tabular}
\end{table}

The main argument about media selection according to media richness theory is that certain communication media are more suitable for certain task depending on the richness of the media and the level of uncertainty and equivocality of the message. A richer media is preferred for high equivocal tasks while leaner media are more suitable for tasks with low equivocality \citep{daft1987}.

\subsubsection{Media Synchronicity Theory}

\begin{itemize}
\item What's the theory all about?
\item What does the theory say about communication media
\end{itemize}

\subsubsection{Media Naturalness Theory}

\begin{itemize}
\item What's the theory all about?
\item What does the theory say about communication media
\end{itemize}

As previously noted, media richness hypothesis has been widely critiziced. As a reponse, Ned Kock proposed an alternative hypothesis of media neutralness to answer to the criticism faced by media richness theory.

Media richness theory was built around the hypothesis that different communication media can be placed on a line where on the other end of the line are the "rich" media and on the other end are the "lean" media. \citep{daft1986}

Media naturalness theory takes a different angle to the problem and start looking at it from the evolution point-of-view. The essential argument of the media naturalness theory is that modern electronical communication media has evolved a lot faster than human species. Thus, modern humans' brains are not optimally adapter for current e-communication technologies. \citep{kock2005}

According to Kock more than 99\% of our evolutionary cycle humans have relied on co-located and synchronous forms of communication. Facial expressions, body language and sounds, including speech have carried an important role in the communication. The muscles of human face have developed to form a complex web that allows us to use rich and expressive facial expressions. Also, there are evidence that the morphology of the human ear suggests a specialized design to decode speech. \citep{kock2005}

Since the e-communication tools have lower capability to transfer these natural elements of human communication the e-communication tools are less natural according to the media naturalness theory in comparison to the face-to-face communication. In fact, media naturalness theory states, the a communication tool with less (or more) natural elements than face-to-face communication is less natural than face-to-face, which is the most natural communication method. \citep{kock2005} \citep{kock2004}

\subsubsection{Media Fitness Theory}

Media Fitness Theory is another theory which tries to address to the mismatch between the previously formed theories and the empirical evidence of media selection. The theory is influenced by Media Richness by \citet{daft1986} Theory and Social Influence Perspectives by Fulk et al. \textbf{LÄHDE}.

The main purpose of Media Fitness Theory is to try to answer the simple question: why choose this medium but not that one \citep{higa2007}. The hypotesis of the theory is that the selection is done because other media are better fit that another. Thus, the theory of media fitness is proposed as: media selection is decided by the fitness of the media with the communication task needs, the communication user and user group, and the supporting environment in which the media being utilized \citep{higa2007}.

MFT defines the fitness of the media by enumerating 14 properties related to fitness and group them into three groups. 

Properties in group I are requirements for the media. Group I properties are closely related to properties defined by MRT and MST. The properties of this are response time, security, sharing, retrieval, multiparty and expressive power. The properties in this group are listed and described in table \textbf{LÄHDE}

\textbf{Tähän se taulukko!}

Properties in group II are properties of communication participants, meaning the user and the user group using the particular communication media. The properties in this group are listed and described in table \textbf{LÄHDE}

\textbf{Tähän se taulukko!}

The last group III are the limitations set by the environment in which the communication occurs. The properties in this group are listed and described in table \textbf{LÄHDE}

\textbf{Tähän se taulukko!}



\subsection{The nature of feedback communication}

Etsitään vastauksia mm. seuraaviin kysymyksiin:

\begin{itemize}
\item Millaisia erityispiirteitä palautteen antamisella on vrt. muu kommunikaatio
\item Miten nämä erityispiirteet vaikuttavat siihen, millaisia ominaisuuksia kommunikaatiovälineen tulisi tukea.
\end{itemize}

\subsubsection{Feedback communication according to MRT}

\begin{itemize}
\item According to MRT, communication media for feedback conversation should be properties A, B and C
\end{itemize}

\subsubsection{Feedback communication according to MST}

\begin{itemize}
\item According to MST, communication media for feedback conversation should be properties A, B and C
\end{itemize}

\subsubsection{Feedback communication according to MNT}

\begin{itemize}
\item According to MNT, communication media for feedback conversation should be properties A, B and C
\end{itemize}

\clearpage

\section{Research question}

\subsection{Objectives}

The communication is an essetial part of agile software development. More over, the feedback given by the customer to the development team is in a crucial role in order to make the project succesful. However, the tools the customers are using to give the feedback have not developed much further. My strong assumption is that most of the feedback is still given via traditional communication tools such as email, phone or face-to-face.

With this thesis I want to build and validate a new kind of feedback tool prototype which could boost the amount and quality of the feedback provided from software project customers to the developers. The objective is to generate understanding of a properties that are essential for feedback communication tools. This knolegde can be utilized in the future for new feedback tool development.

\subsection{Scope}

The focus in this thesis is on feedback communication. A lot of research exists about communication in agile software projects. I believe that customer communication in software projects is a wide subject that requires focused research on each subsubject. Different communication situations require different interactions between customer and the development team. 

This thesis concentrates on software projects and especially on agile software projects. The reason for this is that agile software projects require a lot more collaboration and feedback compared to waterfall style projects where the emphasis is on contract negotiations \textbf{LÄHDE}. In agile software projects the goal of the project is only vaguely defined in the beginning of the project. With extensive communication and collaboration between customer and the developer the goal of the project is crystallized while the project goes on. Thus, it makes sense to scope the thesis only to agile software projects. 

\subsection{Research question}

With this thesis I want to answer to the following question:

\begin{itemize}
\item The research question of this dissertation is \textit{what are properties that make a communication tool effective for giving and receiving feedback in a agile software projects?}
\end{itemize}

\clearpage

\section{Methods}

\subsection{Literature review}

A literature review is done in order to gather theoretical understanding about properties of effective communication media. In this paper the following communication media theories are included: Media Richness Theory, Media Synchronicity Theory and Media Neutralness Theory. 

Media Richness Theory was selected because blah blah...

Media Synchronicity Theory was selected because blah blah...

Media Neutralness Theory was selected because blah blah...

Other communication media theories, such as blah blah was excluded. The reason for exclusion was blah blah...

\begin{itemize}
\item What were the theories to be included?
\item Why those were included?
\item What other pieces of literature was used?
\item What were the inclusion/exclusion criteria for the other pieces of literature?
\end{itemize}

\subsection{Emprical study - Building a prototype for giving feedback}

\begin{itemize}
\item What's Hannotaatio?
\item What kind of properties does Hannotaatio have, from communication media point-of-view
\item What kind of properties are missing from Hannotaatio, from communication media point-of-view
\end{itemize}

\subsection{Validation prototype with qualitative research methods}

Katso apua kirjasta Qualitative Methods: \citep{gummesson1999}

\begin{comment}
- miksi qualitatiivinen menetelmä?
  - miksei quantitatiivinen?
  
  - yksi kvantitatiivinen vaihtoehto: mittaa palautteen määrä ennen Hannotaatiota ja jälkeen
    - ei kerro palautteen laadusta mitään
  - ongelma: Osa käyttänyt Hannotaatiota aika kauan sitten. Mennettä muistellaan aina nykyisten lasien läpi. (Silverman p. 192)
  - 
  
\end{comment}

\textbf{TODO: Kielioppitarkastus}

Kvalitatiivisen tutkimuksen ongelmat!

Based on the theoretical background the implemented prototype Hannotaatio should be suitable to give and receive feedback in software projects. However, this hypothesis has to be validated with empirical evidence. 

In this dissertation qualitative methods were used to validate the prototype. Quantitative methods were also considered, but qualitative method was preferred because of several reasons.

One possible quantitative method would have been to measure the amount of feedback a development team received before and after the use of Hannotaatio feedback tool. This approach would have given a statically reliable proof whether the tool enables feedback conversation between customer and the development team. However, the number of feedback given does not directly imply the value of the given feedback. By examining only the amount of feedback does not tell anything about the quality of the given feedback and the overall value of the feedback. 

A structured questionnaire was another considered quantitative method. A structured survey could have overcome the problem of measuring the sheer number of feedback. With a survey it could have been possible to ask questions about the quality of feedback and the perceived value of the feedback. There are also a number of statical analysis methods which could have been used with structured survey.

However, structured questionnaire have some drawbacks, which make it an unsuitable method for the particular case. First, surveys can give answers to questions that are known when the survey is created but they allow very poorly new questions and ideas to arise. In this particular case I am especially interested to hear new ideas how to improve the prototype to make it even better tool for feedback. Second, questionnaires require a great number of answers to form a statically reliable sample. However, the amount of available contact information of Hannotaatio is very limited and thus a reliable sample for quantitative methods could not have been formed.

\subsubsection{Data collection with semi-structured interviews}

THe ultimate purpose of the semi-structured interviews is to validate whether the properties of communication media implemented in Hannotaatio support feedback communication as the theorethical background suggests. 

\citep{silverman2009doing}

\begin{comment}

- mikä on semi-structured interview
- miksi semi-structured
  - miksei täysin avoin?
  - miksei structured?
- mikä on haastattelujen ultimate tarkoitus
  - miksi haastattelut pidetään?
- miksi face-to-face
  - mutta kuitenkin yksi oli sähköpostilla
- millä kielellä, miksi?

- eksploratiivinen, ei tiedetä tarkasti mitä haetaan.
- miksei case study tutkimus?
  - case study voisi väärentää, tiedetään, että tästä tehdään tutkimusta
- ei välttämättä laajalti yleistettäviä tuloksia
  - sen sijaan saadaan osviittaa ja uusia ehdotuksia
- ei arvioida käytettävyyttä, ei arvioida itse tuotetta
  - vaara: prototyyppi on hyvin valmis, saattaa viedä huomiota liikaa tuotteeseen, ei ajatukseen sen takana
  
- miten haastateltavat etsittiin ja valittiin
- miksi litteröitiin / miksei litteroitu

- mitä eri mahdollisuuksia on qualitatiivisen datan analysoitiin
- mikä valittiin, miksi?

- data-analyysi: grounded theory, narrative, conversation, discourse analysis

\end{comment}

The data for prototype validation was gathered by conducting semi-structured interview with people who have worked in agile software projects and use Hannotaatio in those projects.

Semi-structured interviews was selected for a research method because number of reasons. As the result of the desired result yet is unknown, it makes sense to set the stage for the interview and let the discussion flow. According to Mason using semi-structured interviews allow even unexpected themes to emerge. However, because the general themes of the interview are known beforehand, semi-structured interviews allow interviewer to ensure that the relevant contexts are brought into focus so that situated knowledge can be produced \citep{mason2004}

\subsubsection{Finding interviewees}

Hannotaatio is a publicly available tool, which can be used without registration. The ability to use the tool without registration is friendly for the users but it made contacting the users extremely difficult because the contact information of the users was not available. Providing a email address is only optional and thus the amount of email addresses in Hannotaatio's database is very limited. All of the users who had provided an email address were contacted and asked for an interview.

Hannotaatio's database of users' email addresses contains only email addresses to be used as a notification emails to the developers when a new feedback has been sent. Thus the people who were contacted were all developers or other persons who were receiving the feedback via Hannotaatio, not sending it. For the research purposes it would have been valuable to interview both roles of the feedback communication, feedback provider and feedback receiver. However, the contacted people were not very willing to give contact information of their customers. Thus, only the feedback receivers were interviewed.

\subsubsection{Preparing interviews}

Before the interviews a structure for the interviews was created. The purpose of this structure was to create a baseline, which was loosely followed. As the interviews were semi-structured, the baseline structure left a lot of open space for new themes to arise.

A practice interview was conducted before the first recorded interview to test the content and length of the interview structure. After the practice interview minor changes to the interview structure was made.

\subsubsection{Conducting interviews}

7 people were interviewed in total. 6 of them were interviewed face-to-face and 1 was interviewed via email. Face-to-face was preferred because it allows interviewer to react on the response and possibly ask follow up questions. One of the seven interviewees was interviewed via email due to time restrictions and physical distance of the interviewee.

The interviews started with a warm-up questions including basic information and job title of the interviewee and general description about the project where Hannotaatio was used. The middle section of the interview included more detailed questions about Hannotaatio as a feedback tool in a software project. The interviews ended with a open question where the interviewee was able to tell anything that she felt was missing.

All interviews were recorded. The language of the interviews was Finnish which the native language for the interviewer and for all the interviewees.

\subsubsection{Analysing interviews}

The data analysis process started in parallel with interviews. The analysis process loosely follows coding practice. As the interviews were recorded the tapes were listened and the most important themes were written down. 

The purpose of the analysis phase was to firstly find out a common themes shared between interviews and secondly find out interesting view points from individual interviewees.

\clearpage

\section{Results}

\subsection{Hannotaatio - A visual website feedback tool}

Tähän yleiskuvaus siitä, mikä on Hannotaatio. Vastaa kysymykseen "millainen prototyypistä sitten loppujen lopuksi tuli?". Käytä Screenshotteja

\subsection{Implemented features}

\subsubsection{Easy installation}

An easy installation of Hannotaatio was one of the main goals for the product. Complex installation process can be a show stopping barrier for users who would like to try new communication tools but are not willing to investment too much effort for the introduction of the tool. Because of that, Hannotaatio's installation process is implemented to be as easy and fast as possible.

The installation of Hannotaatio requires minimal amount of coding and configuration. In the simplest case user does not have to do anything else that copy the seven line code snippet provided in Hannotaatio website to user's own site \citep{hannotaatio}.

For advanced users, there is a possibility to create an API key. The purpose of the API is to collect email addresses of the users show that they can be informed about upcoming updates and down-times.

There is also a possibility for advanced users change the default site capturing settings, e.g. turn on the image capturing. This allows smooth use of Hannotaatio even with private websites that are protected by passwords or firewalls.

\subsubsection{Initiating feedback process}

Before customer can start giving the feedback, two things have to happen. First, customer has to see and try out the production from which the feedback will be given. Second, customer has to initiate the selected communication channel. With the traditional communication tools such as email or telephone, the initiation process requires opening email software and creating a new email message or calling to the feedback receiver.

In Hannotaatio a lot of work has been done to make the initiation of the feedback process as easy as possible. The solution to enable easy initiation of feedback communication channel was to add an "I love feedback" button to the upper right corner of the website from which the feedback will be given. This way the gap between the feedback subject and communication tool is minimal.

\begin{figure}[htb]
\begin{center}
\includegraphics[width=1.0\textwidth]{initiate_feedback.png}
\end{center}
\caption{"I love feedback" button is added to the upper right corner of the website}
\end{figure}

When user presses the "I love feedback" button, a screen capture is taken from the website. After that, user is redirected to an editor, where user is able to draw on top of the captured website.

\subsubsection{Drawing tools}

In Hannotaatio, there are couple of tools for user to draw the feedback on top of the website. The number of tools have been kept minimum on purpose to make the application extremely simple to user.

The available drawing tools are pointing arrow, rectangle and text box. Also, the color of the drawing can be changes between dark and light color scheme. This allows user to draw on top of either light or dark websites.

In requirements gathering phase it was identified that the two most important functions of drawing tools are pointing, highlighting an area and leaving textual note. The three implemented drawing tools allow all the three. However, other drawing tools such as freehand drawing tool or circle drawing tool was left unimplemented, because they we're not critical tools to accomplish the desired functions of pointing, highlighting and leaving a note.

\begin{figure}[htb]
\begin{center}
\includegraphics[width=1.0\textwidth]{drawing_tools_annotated_crop.png}
\end{center}
\caption{Hannotaatio toolbar}
\end{figure}

\subsubsection{Sharing the feedback with the team}

When the customer has drawn all the feedback with the available drawing tools, the first step to share the feedback with the team is to publish the feedback by pressing Publish button. When the drawn feedback is published no further modification can be made.

After publishing, user is given a secure URL, which she can share with the team for example via email. The secure URL is randomly generated UUID and it is long enough so that it is impossible to guess. That makes it secure even though viewing the feedback does not require password or any other user credential.

Optionally, if the team has set predefined notification email addresses, a notification email is sent to the team. This happens right after the feedback is published. If notification emails are used, customer does not have to share the secure URL with the team separately.

\subsubsection{Viewing the feedback}

After the drawn feedback has been published by the customer, the development team receives a notification email with the secure URL to the newly drawn feedback or the team receives the secure URL from the customer via email.

The team can now access to the published feedback. Besides seeing the drawn feedback the team can also see when the feedback was given and which browser and operation system was used. Additionally, team is able to access the original site from which the feedback was given by clicking "Go to original page". Also, team or whoever has access to the secure URL and delete the feedback by pressing "Delete" button.

\begin{figure}[htb]
\begin{center}
\includegraphics[width=1.0\textwidth]{published_feedback.png}
\end{center}
\caption{Published feedback}
\end{figure}

\subsection{Hannotaatio MRT}

Vastaa kysymykseen: Millainen kommunikaatiotyökalu Hannotaatio on MRT:n mukaan. Sopiiko palautteenantoon?

\subsection{Hannotaatio MST}

As noted in the previous sections, Media Synchronicity Theory identifies the following properties of a communication media: Transmission velocity, parallelism, natural symbol set, rehearsability and reprocessability.

In Hannotaatio the \textbf{transmission velocity} is low. When a developer team has something to show to the customer email is commonly used to notify customer about the new version from which she can give feedback. After the customer has received the notification from a developer team she browses to the site, gives feedback with Hannotaatio and shares the secure URL with the team via email.

The transmission speed of email is instant, but because getting a response to email adds some delay, email is considered to have low transmission velocity. Because there are at least two email send-receive cycles involved in one feedback which is given with Hannotaatio, it can be argued that the transmission velocity for Hannotaatio is rather low.

In Hannotaatio there is a possibility to use notification emails. If notification emails are used, the notification is sent to the team automatically right after the customer has published the feedback. This feature slightly improves the transmission speed because it eliminates one manual email sending from the whole feedback process.

Hannotaatio supports high \textbf{parallelism}. Because giving feedback with Hannotaatio does not require shared time and location with the feedback receiving team, customer can have many simultaneous feedback conversations at the same time. In the other words this means that customer can give feedback with Hannotaatio at the same time when shes chatting with the team with an instant messaging tool. However, it must be noted that drawing the feedback requires some concentration from the customer, so even if it is possible to have multiple conversions at the same time, it may not be very pleasent.

Hannotaatio supports also high \textbf{rehearsability}. Because the feedback is not transmitted to the developer team before customer chooses to publish it, customer has the ability to fine-tune the feedback drawing as long as she want. For feedback conversation this property of Hannotaatio is important, so that customer can fine-tune the message to be as clear and understandable as possible. Because feedback can be sometimes negative, it is also good that the customer has the ability to choose the wording carefully.

Hannotaatio supports high \textbf{reprocessability}. After the customer has shared the secure URL to the development team the team can come back to the URL which contains the message as many times as needed. From the feedback point-of-view this property of the tool is extremely important since the team may not have time to react to the feedback immediately. For example, in agile development, it might take some weeks before the team reacts to the feedback, it the team decides to do it in the next iteration. In this case it is important to be able to recap what was the feedback all about.

The naturalness of the \textbf{symbol set} in Hannotaatio can be argued to be medium. Visual message is more natural than for example written message. Because Hannotaatio supports visual encoding of the message (annotated screenshot) it has a more natural symbol set than e.g. plain text email, which (attachments excluded) supports only written message.

However, even though the message in Hannotaatio can be visually encoded, Hannotaatio misses for example vocal tones which can be transferred with for example telephone and physical gestures which can be transferred with for example video conferencing system or face-to-face. Thus it can be argued, that Hannotaatio does not have the most natural symbol set, instead medium level of naturalness.

\begin{table}[!h]
% increase table row spacing, adjust to taste
\renewcommand{\arraystretch}{1.3}
% if using array.sty, it might be a good idea to tweak the value of
% \extrarowheight as needed to properly center the text within the cells
\caption{Media capabilities and their importance for feedback}
\label{table:capabilities}
\centering
% Some packages, such as MDW tools, offer better commands for making tables
% than the plain LaTeX2e tabular which is used here.
\begin{tabular}{|p{4cm}|p{7cm}|p{3cm}|}
\hline
\textbf{Media \newline capability} & \textbf{Description} & \textbf{Support in Hannotaatio}\\
\hline
Transmission \newline velocity & The speed at which the information is transported from an individual to another & Low\\
\hline
Parallelism & Capability for multiple parallel communication sessions & High\\
\hline
Natural symbol set & Diversity of symbols which allows information encoding. Natural symbols are vocal tones and physical gestures etc. & Medium\\
\hline
Rehearsability & The ability to fine tune the message before sending it & High\\
\hline
Reprocessability & The possibility to reprocess the transmitted message & High\\
\hline
\end{tabular}
\end{table}

\subsection{Hannotaatio MNT}

Media Naturalness Theory emphasizes communication tools that are as close to face-to-face communication as possible. REF. The features that make communication tool natural are they

\subsection{Hannotaatio, theoretical conclusion}

Edellisissä luvuissa käsiteltiin kutakin teoriaa ja Hannotaatiota erikseen. Tässä luvussa vedetään yhteen.

\subsection{Results of the semi-structured interviews}

Tässä luvussa kerrotaan haastattelujen tulokset.

\begin{itemize}
\item The interviews revealed that property A was very good
\item The interviews revealed that property B was missing and it would have been beneficial.
\end{itemize}

\begin{comment}
Tässä osassa esitetään tulokset ja vastataan tutkielman alussa
esitettyihin tutkimuskysymyksiin. Tieteellisen kirjoitelman
arvo mitataan tässä osassa esitettyjen tulosten perusteella. 

%% Huomaa seuraavassa kappaleessa lainausmerkkien ulkopuolella piste, 
%% koska piste ei lopeta lainattua tekstinpätkää.
%% Jos lainattu tekstinpätkä loppuu välimerkkiin, tulee välimerkki
%% lainausmerkkien sisälle: 
%% "Et tu, Brute?" sanoi Caesar kuollessaan.
Tutkimustuloksien merkitystä on aina syytä arvioida ja tarkastella
kriittisesti.  Joskus tarkastelu voi olla tässä osassa, mutta se
voidaan myös jättää viimeiseen osaan, jolloin viimeisen osan nimeksi
tulee >>Tarkastelu>>. Tutkimustulosten merkitystä voi arvioida myös
>>Johtopäätökset>>-otsikon alla viimeisessä osassa. 

Tässä osassa on syytä myös arvioida tutkimustulosten luotettavuutta.
Jos tutkimustulosten merkitystä arvioidaan >>Tarkastelu>>-osassa,
voi luotettavuuden arviointi olla myös siellä. 

\end{comment}

% \clearpage

\section{Discussion}

\begin{enumerate}
\item What are the weak points of this study?
\item What could be studied in the future?
\end{enumerate}

% \clearpage


\begin{comment}
Tässä osassa selvitetään, mitä tutkimuksen kohteena olevasta
aiheesta tiedetään entuudestaan. Selvityksen tulee kattaa
tasapainoisesti koko tutkimuskenttä. 

Kun opinnäytetyötä kirjoitetaan, on noudatettava 
ohjeita, jotka koskevat opinnäytteen rakennetta,
käytäntöjä, muotoseikkoja sekä ulkoasua. Esitellään näitä
ohjeita tarkemmin.
\end{comment}

%% Osan hienojaottelua alaosiin, eikä välttämättä edes tarpeen,
%% tässä vain esimerkkinä. Käytä harkintasi mukaan
%% osan jaottelua, joskus alaotsikot selventävät asioita ja
%% joskus vain sirpaloittavat tarpeettomasti tekstiä.
%%  Jaottelu menee seuraavasti:
%% \section{osan otsikko} 
%% \subsection{alaotsikko}
%% \subsubsection{ala-alaotsikko}
%% Tätä pitemälle ei pidä jaotella. 
%%
%% Three levels of hierarchy in sectioning should be enough

\begin{comment}
\subsection*{Rakenne}

Opinnäytteen rakenteen tulee olla hyvän tieteellisen
kirjoittamisen käytännön mukainen ja sisältää vähintään seuraavat
osat:

\begin{enumerate}
\item Nimiölehti
\item Tiivistelmä
\item Sisällysluettelo
\item Symboli- ja lyhenneluettelo
\item \label{a} Johdanto
%% Tässä alla on esimerkki lainausmerkkien käytöstä. Suomalaisen tekstin
%% lainausmerkit eivät mene oikein latexissa (tai monissa muissakaan
%% julkaisujärjestelmissä) kun käytetään
%% "-merkkiä, koska latex käyttää amerikkalaista lainausmerkkien
%% tulostustapaa. Vaihtoehtona voi käyttää kulmalainausmerkkejä, jotka
%% myös tulostuvat oikein.
\item  Aikaisempi tutkimus. Työn luonteen niin vaatiessa otsikko voi olla myös
        >>Teoreettinen tausta>>  tai näiden otsikoiden yhdistelmä.
\item Tutkimusaineisto ja -menetelmät %% yhdysmerkki - eli tavuviiva. 
\item Tulokset
\item \label{o} Tarkastelu. Työn luonteen niin vaatiessa otsikko voi
      olla myös >>Johtopäätökset>> tai >>Yhteenveto>> 
      tai edellä mainittujen otsikoiden yhdistelmä.
\item Lähteet
\item Liitteet.
\end{enumerate}

Tiivistelmän ja symboli- sekä lyhenneluetteloiden 
väliin voi sijoittaa halutessaan esipuheen.  

Työn osat \ref{a}-\ref{o} muodostavat \textit{tekstiosan.}  Työn
yksittäisiä osia voidaan jakaa alaotsikoilla alaosiin, joita ei ole
yllä esitetty. Alaotsikoiden käyttäminen selventää parhaimmillaan
tekstiä, ja pahimmillaan sirpaloittaa sitä.  Sirpaloitumista voi estää
huolehtimalla siitä, että samalla sivulla ei esiinny useampaa
alaotsikkoa.  Tekstin jäsentelyssä on yleensä ongelmia, jos osassa on
vain yksi alaosa, tai kirjoittaja joutuu käyttämään useampaa kuin
kahta tasoa (osa ja alaosat): alaosien alaosat ovat harvoin tarpeen.
\subsection*{Sivut ja kirjaintyypit}

Opinnäytteen tulee olla kirjoitettu koneella tai
tekstinkäsittelyohjelmalla yksipuolisesti A4-kokoiselle paperille.
Kandidaatintyön tekstiosan sopiva pituus on noin 15--20 sivua ja
diplomityön noin 60 sivua. Työtä ei ole syytä tarpeettomasti pidentää.

Opinnäytteen tekstiosan kirjaintyypin tulee olla antiikva eli
%% esimerkki pakkotavutuksesta; "serif-tyyppinen" on tavutuksen kannalta
%% hankala, joten pakkotavutetaan se. 
serif\--tyyp\-pi\-nen ja lisäksi kursivoimaton, lihavoimaton sekä kooltaan 12
pistettä (kuten tässä esityksessä). Groteskeja eli \textsf{Sans
  serif}-tyyppisiä kirjaintyyppejä (kuten Helvetica tai Arial) ei saa
käyttää varsinaisessa tekstissä, mutta otsikoissa näitä voidaan
käyttää.  Otsikoissa voidaan käyttää kooltaan edellä mainittua
suurempaa kirjaintyyppiä sekä tyylikeinoja, kuten lihavointia tai
kursivointia.  Tekstissä samantasoisten otsikoiden on kuitenkin oltava
tyyliltään ja kirjainlajeiltaan yhteneväisiä.
%% Esimerkki taulukosta
\begin{table}[htb]
%% Taulukon teksti
\caption{Taulukoissa ja kuvissa kirjaintyypin voi valita
tarkoituksenmukaisesti, mutta kuva- ja taulukkoteksteissä tulee
käyttää samaa kirjaintyyppiä kuin varsinaisessa tekstissä. 
Huomaa taulukon numeroinnin sijoittuminen taulukon yläpuolelle. \label{taulukko1}}
\begin{center}
\fbox{
\begin{tabular}{c|l|r}
\textbf{A} & 1 & $e^{j \omega t}$ \\ \hline
\textsf{B} & 2 & ${\mathfrak R}(c)$ \\ \hline
\texttt{C} & 3 & $ a \in \mathbb{A}$  
\end{tabular}
}
\end{center}
\end{table}

Opinnäytteen vasen marginaali (sidonnan puoli) on
35~mm % tässä ~ muodostaa ns. yhdistävän välilyönnin
ja oikea 25~mm. Ylämarginaali on 25~mm. Leipätekstin korkeus on
enimmillään 230mm. Tämän opinnäytepohjan marginaalien pitäisi olla
paperille tulostettuna oikein, mutta tulostimesta ja paperista
riippuen voi esiintyä yhden tai kahden millimetrin suuruisia eroja.
%% Jos käännät tämän tekstin pdflatex-komennolla ja tulostat sen katselu-
%% ohjelmasta, toteat todennäköisesti em. mittojen poikkeavan enemmän
%% kuin 1-2 mm. 
%% Tämä on seurausta pdf-tiedoston erilaisesta kirjaintyyppimäärityksestä.
%% Korkeatasoista painotyötä varten käytä vain latex-komentoa ja 
%% tulosta postscript-muotoon käännetystä tiedostosta. 
\subsection*{Asemointi}

%% Muutos vanhaan ohjeeseen verrattuna: aikaisemmassa ohjeessa
%% kehotettiin käyttämään vasensuora-asettelua, mutta tässä
%% ohjeessa ollaan luovuttu tuosta vaatimuksesta ja siirrytty
%% huoliteltumpaan, painotuotteenomaisempaan suuntaan.  
Tekstiosan tekstissä käytetään kappaleiden erottamiseen sisennystä,
mutta ensimmäistä otsikon, väliotsikon tai muun katkon jälkeistä
kappaletta ei sisennetä. Jos kuva tai muu katko tulee kappaleiden
väliin, suositellaan katkon jälkeisen kappaleen sisentämistä.

Mikäli oikea reuna halutaan tasata, tulee käyttää tavutusta ja lisäksi
tarkistaa, ettei tekstiin jää lukemista häiritseviä pitkiä sanavälejä. Jos
käytät opinnäytteen tekemisessä \LaTeX-järjestelmää, 
tämä asia hoituu automaattisest.

Opinnäytteen riviväli on 1, mikä on myös tämän opinnäytepohjan käytäntö. 
Kappaleiden tulee yleensä olla ainakin kolmen rivin pituisia, mutta
myös liian pitkiä kappaleita tulee välttää.  Tässä opinnäytepohjassa
ei tekstin luonteen vuoksi voida täysin toteuttaa kappaleen pituutta koskevia
vaatimuksia.

Yksittäisiä, kappaleen päättäviä tai aloittavia rivejä sivun alussa
tai lopussa on vältettävä koko työssä, myös luetteloissa ja
liitteissä.

\subsection*{Numerointi}

Opinnäytteen jokainen osa alkaa uudelta sivulta. Alaosa aloittaa uuden
sivun vain edellisen sivun täytyttyä.

Työn osat numeroidaan siten, että johdanto on ensimmäinen numeroitava
osa. Osien numeroinnissa käytetään arabialaisia numeroita.

Nimiölehti, tiivistelmä, esipuhe, sisällysluettelo ja symboli- ja
lyhenneluettelo numeroidaan esipuheesta tai tämän puuttuessa 
ensimmäiseltä luettelosivulta alkaen roomalaisin numeroin.

Sivunumerointi alkaa toiselta varsinaiselta tekstisivulta, ja 
sivunumeroinnissa käytetään arabialaisia numeroita.

Lähdeluettelo alkaa uudelta sivulta. Lähdeluettelon sivunumerointi 
jatkuu viimeisestä tekstisivusta.

Jokainen liite alkaa uudelta sivulta. Liitteiden sivunumerointi
jatkuu viimeisestä lähdeluettelon sivusta.

Sivunumero sijoitetaan sivun yläreunaan.

Matemaattiset kaavat numeroidaan arabialaisin
numeroin. Kaavanumerointi ei saa katketa osien välissä (eikä niin
tapahdukaan, jos käytät tätä opinnäytepohjaa). Kaikkia kaavoja ei tarvitse
numeroida, vaan kirjoittaja voi käyttää harkintaa numeroinnin
tarpeellisuudessa.  Liitteissä olevat kaavat numeroidaan siten, että
liitteen ajatellaan muodostavan numeroinnin kannalta itsenäisen ja
yhtenäisen kokonaisuuden. Kaavan numero sijoitetaan oikealle puolelle
alla olevan esimerkin mukaisesti
\begin{equation}
D(xy) = (Dx)y + x(Dy),  \hspace{3em} x,y \in \mathbb{A}.
\end{equation}
%% Kaavojen jälkeen ei yleensä laiteta sisennystä. 
Kaikki kuvat ja taulukot numeroidaan erillisen juoksevan numeroinnin
mukaisesti kuten taulukosta \ref{taulukko1} ja kuvasta \ref{kuva1} käy
ilmi.  Liitteissä olevat kuvat ja taulukot numeroidaan siten, että
liitteen ajatellaan muodostavan numeroinnin kannalta itsenäisen ja
yhtenäisen kokonaisuuden. Liitteissä \ref{LiiteA} ja \ref{LiiteB} on
esimerkkejä kaavojen (kaavat \ref{liitekaava1}--\ref{liitekaava2} tai
kaavat \ref{liitekaava3}--\ref{liitekaava4}), kuvien (kuva
\ref{liitekuva}) ja taulukoiden (taulukko \ref{liitetaulukko})
numeroimisesta.  Liitteet numeroidaan suuraakkosin (esimerkiksi Liite
A, Liite B tai pelkästään A, B).
%% Tässä esimerkki kuva1.pdf -nimisen tiedoston tuomisesta kuvaksi.
%% Komento \inclugraphics[parametrit]{argumentti} tuo kuvan.
%% Komento \centering pakottaa kuvan keskelle. 
%% Komento \caption luo kuvatekstin ja sen numeroinnin
%% Parametrit htb pakottavat kuvan suunnilleen siihen 
%% kohtaan, missä se esiintyy tekstin lähdekoodissa
\begin{figure}[htb]
\centering \includegraphics[height=5cm]{kuva1}
\caption{Tämä on esimerkki numeroidusta kuvatekstistä. \label{kuva1}}
\end{figure}

\subsection*{Lähdeviittausten käyttö} 

\begin{comment}

Lähdeviittaukset tulee tehdä huolellisesti ja johdonmukaisesti
numeroviitejärjestelmän mukaisesti. Numeroviitteet järjestetään
lähdeluetteloon viittausjärjestykseen, mutta jos lähdeluettelo
on hyvin laaja (useita sivuja), järjestetään viitteet pääsanan 
mukaiseen aakkosjärjestykseen. Alaviitejärjestelmää
\footnote{Myöskään alaviitteenä olevia kommentteja \underline{ei} suositella
käytettäviksi.} ei käytetä. 

Viitteen sijoittelussa noudatetaan seuraavia sääntöjä:
Jos viite kohdistuu vain yhteen virkkeeseen tai virkkeen 
osaan, viite \cite{Kauranen} sijoitetaan virkkeen sisään ennen virkettä
päättävää pistettä. Jos taas viite koskee tekstin useampaa
virkettä tai kokonaista kappaletta, sijoitetaan viite kappaleen loppuun 
pisteen jälkeen. \cite{Kauranen} 

\subsection*{Lähdeluettelo} 

Lähdeluettelossa esiintyy tavallisesti seuraavassa esitettäviä
lähteitä, joista on numeroviitejärjestelmässä ilmoitettava
asianomaisessa kohdassa vaaditut tiedot.

%% Esimerkki korostamisesta. Lihavoinnin sijasta on tyylikkäämpää
%% ja luettavampaa käyttää kursiivia.
\textit{Kirjasta} ilmoitetaan seuraavat tiedot:

\begin{itemize}
\item[--]tekijät 
\item[--]julkaisun nimi
\item[--]painos, jos useita
\item[--]kustannuspaikka
\item[--]julkaisija tai kustantaja
\item[--]julkaisuaika
\item[--]mahdollinen sarjamerkintö. 
\end{itemize}

Viitteet \cite{Kauranen}--\cite{Koblitz} ovat esimerkkejä kirjan
esittämisestä lähdeluettelossa. Viite \cite[s.\ 83--124]{Koblitz} on
esimerkki lähdeluettelossa esiintyvän kirjan tiettyjen sivujen
esittämisestä tekstissä.

\textit{Artikkelista} kausijulkaisussa ilmoitetaan seuraavat tiedot:

\begin{itemize}

\item[--]tekijät
\item[--]artikkelin nimi
\item[--]kausijulkaisun nimi
\item[--]julkaisuvuosi
\item[--]kausijulkaisun volyymi tai ilmestymisvuosi
\item[--]kausijulkaisun numero
\item[--]sivut, joilla artikkeli on.
\end{itemize}

Viitteet \cite{bcs}--\cite{Deschamps} ovat esimerkkejä artikkelin
esittämisestä lähdeluettelossa.

\textit{Kokoomateoksen luvusta tai osasta} ilmoitetaan seuraavat tiedot:

\begin{itemize}
\item[--]luvun tai osan tekijät
\item[--]luvun tai osan nimi
\item[--]maininta >>Teoksessa>>
\item[--]koko teoksen toimittajat sekä maininta >>(toim.)>>
\item[--]koko teoksen tai konferenssin nimi
\item[--]konferenssiesitelmän kyseessä ollessa sen pitopaikka ja -aika
\item[--]painos, jos useita
\item[--]kustannuspaikka
\item[--]julkaisija tai kustantaja, jos aihetta tämän ilmoittamiseen on
\item[--]julkaisuaika
\item[--]sivut, joilla luku tai osa on 
\item[--]mahdollinen sarjamerkintä.
\end{itemize}

Viitteet \cite{Sihvola}--\cite{Lindblom} ovat esimerkkejä
kokoomateoksen luvun tai osan esittämisestä lähdeluettelossa. 

\textit{Opinnäytetyöstä} ilmoitetaan seuraavat tiedot:

\begin{itemize}
\item[--]tekijä
\item[--]työn nimi
\item[--]opinnäytetyön tyyppi
\item[--]oppilaitoksen nimi
\item[--]osaston, laitoksen tai ohjelman nimi
\item[--]oppilaitoksen sijaintipaikka
\item[--]vuosiluku.
\end{itemize}

Viitteet \cite{Miinusmaa}--\cite{Lonnqvist} ovat esimerkkejä
opinnäytteen esittämisestä lähdeluettelossa. 

\textit{Standardista} ilmoitetaan seuraavat tiedot:

\begin{itemize}
\item[--]standardin tunnus ja numero
\item[--]standardin nimi
\item[--]painos, mikäli ei ole ensimmäinen
\item[--]julkaisupaikka
\item[--]julkaisija
\item[--]julkaisuvuosi
\item[--]sivumäärä.
\end{itemize}
Viite \cite{sfs} on esimerkki standardin esittämisestä opinnäytteen
lähdeluettelossa. 

\textit{Haastattelusta} ilmoitetaan seuraavat tiedot:

\begin{itemize}
\item[--]haastatellun henkilön nimi
\item[--]haastatellun henkilön arvo tai asema
\item[--]haastatellun henkilön edustama organisaatio
\item[--]organisaation osoite
\item[--]maininta siitä, että kyseessä on haastattelu ja haastattelun
päivämäärä. 
\end{itemize}

Viite \cite{haastattelu} on esimerkki 
haastattelun esittämisestä lähdeluettelossa.

Osa sähköisessä muodossa olevista artikkeleista on saatavissa myös
painettuina. \textit{Vain verkosta saatavissa olevasta artikkelista} esitetään
seuraavat tiedot:

\begin{itemize}
\item[--]tekijät
\item[--]artikkelin nimi
\item[--]kausijulkaisun nimi
\item[--]viestintyyppi
\item[--]laitos tai volyymi
\item[--]kausijulkaisun yksittäistä osaa koskeva merkintä tai numero
\item[--]julkaisuvuosi tai maininta >>Päivitetty>> ja päivitysaika
\item[--]maininta >>Viitattu>> ja viittaamisen ajankohta 
\item[--]maininta >>Saatavissa>> ja URL tai 
        maininta >>DOI>> ja DOI-numero (DOI=Digital Object Identifier).
\end{itemize}

Viitteet \cite{Ribeiro}--\cite{kone} ovat esimerkkejä sähköisessä
muodossa olevan artikkelin esittämisestä opinnäytteen
lähdeluettelossa.  Viitteet \cite{Ribeiro} ja \cite{Stieber} ovat
saatavissa sekä painettuna että verkosta, joten viitteiden esitystapa
mukailee painetun artikkelin viitteen esitystapaa, mutta sen lisäksi
kerrotaan julkaisun olevan verkkolehti ja lehden olevan saatavissa
myös painettuna.  Viite \cite{kone} on saatavissa vain verkosta ja
siitä esitetään yllä vaaditut tiedot.

Valitettavasti sähköisessä muodosssa olevasta artikkelista ei ole aina 
saatavissa lai\-tos-, volyymi- tai numerotietoja.

\textit{Sähköisessä muodossa olevasta opinnäytetyöstä} ilmoitetaan
seuraavat tiedot:
 
\begin{itemize}
\item[--]tekijä
\item[--]työn nimi
\item[--]viestintyyppi
\item[--]opinnäytetyön tyyppi
\item[--]oppilaitoksen nimi
\item[--]osaston, laitoksen tai ohjelman nimi
\item[--]oppilaitoksen sijaintipaikka
\item[--]vuosiluku
\item[--]viittamisen ajankohta
\item[--]maininta >>Saatavissa>> ja URL tai 
        maininta >>DOI>> ja DOI-numero.
\end{itemize}

Viite \cite{Adida} on esimerkki sähköisessä muodossa olevan
opinnäytteen esittämisestä lähdeluettelossa.

Viite \cite{viittaaminen} on esimerkki itsenäisen kirjoituksen sisältävästä
verkkosivusta. Tällainen lähde on rinnastettavissa erillisteokseen.
\textit{Verkkosivusta} esitetään tiedot:

\begin{itemize}
\item[--] tekijät
\item[--] otsikko
\item[--] maininta >>Päivitetty>> ja päivitysaika 
\item[--] maininta >>Viitattu>> ja viittaamisen ajankohta
\item[--] Maininta >>Saatavissa>> ja URL.
\end{itemize}

Joskus verkkosivun kirjoitus on jaettu useammalle sivulle, jolloin
lähdeluetteloon kirjataan vain sellainen verkko-osoite, joka koskee
koko kirjoitusta tai sen etusivua, ellei sitten 
todella tarkoiteta kirjoituksen yksittäistä sivua. 

\subsection*{Muuta huomioitavaa lähdeluettelossa}

%% Muutos vanhoihin ohjeisiin koskien kieltä.
Lähdeluettelossa työn ja julkaisun nimi kirjoitetaan alkuperäisessä
muodossaan. Julkaisijan kotipaikka kirjoitetaan alkukielisessä
muodossaan.

Viittamista koskevassa suomalaisessa standardissa
SFS 5342 \cite{sfs} vaaditaan julkaisuista ilmoitettavaksi myös ISBN- tai
ISSN-numerot, mutta näissä opinnäyteohjeissa ei ISBN- ja 
ISSN-numeroita vaadita. 

\end{comment}

%% Lähdeluettelo
\bibliographystyle{plainnat}
\bibliography{ref}

%% Liitteet 
\appendix 

\clearpage

% \addcontentsline{toc}{section}{Appendix A}
\section{Interview Questions\label{appendix:interview_questions}}

Alustus: Teen haastattelun diplomityötäni varten Aalto-yliopiston informaatioverkostojen koulutusohjelmaan. Haastattelulla pyrin selvittämään Hannotaatio palautetyökalun käyttöä sekä palautteen antamista yleisemmin ohjelmistoprojekteissa. Haastattelu nauhoitetaan. 

\begin{enumerate}

\item Titteli, ikä? Saanko käyttää näitä työssäni?

\item Yritys, jossa työskentelet? Saanko tarvittaessa käyttää näitä työssäni?

\item Kerrotko lyhyesti, mitä sinä ja yrityksesi teette?

\item Miten yrityksessänne hoidetaan palautteen anto asiakas-toimittaja suhteessa?

\item Mitkä asiat vaikuttavat siihen, millä tavoin ja millä välineillä palautekommunikaatio hoidetaan?

\item Millaisessa projektissa olette käyttäneet Hannotaatiota?

\item Miten Hannotaation käyttö on muuttanut palautteenantoa, vai onko?
  \begin{enumerate}
    \item Miten paljon palautetta on tullut Hannotaation kautta?
    \item Miksi niin paljon / miksi niin vähän?
  \end{enumerate}


\item Miten paljon arvostat palauttetyökalussa seuraavia ominaisuuksia:
  \begin{enumerate}
    \item Välitön palaute / vastausaika \textit{(MRT, MST, MFT)}
    \item Persoonallisuus \textit{(MRT)}
    \item Ilmaisun monimuotoisuus \textit{(MRT, MST, MFT)}
    \item Ilmaisun luonnollisuus \textit{(MRT, MST, MFT)}
    \item Samanaikaisuus \textit{(MST)}
    \item Mahdollisuus viimeistellä \textif{(MST)}
    \item Mahdollisuus uudelleenprosessointiin \textif{(MST)}
    \item Turvallisuus \textit{(MFT)}
    \item Helppo jaettavuus \textit{(MFT)}
    \item Haku myöhempää käyttöä varten \textit{(MFT)}
    \item Monen käyttäjän yhtäaikainen käyttö \textit{(MFT)}
    \item Hinta \textit{(MFT)}
  \end{enumerate}
  
\item Miten nämä ko. ominaisuudet on toteutettu mielestäsi Hannotaatiossa?
  
\item Millaisia uusia ominaisuuksia kaipaisit Hannotaatioon (jotta siitä tulisi entistä parempi työkalu palautteenantoon)? \textit{MNT}
  \begin{enumerate}
    \item Videokuva palautteen piirtämisestä?
    \item Videokuva + ääniraita
    \item Videokuva + ääniraita + videokuva palautteenantajasta
  \end{enumerate}

\item Onko sinulla vielä jotain muuta jota haluaisit lisätä koskien palautteen antoa yleisesti tai Hannotaatio-työkalua?
  
\end{enumerate}

\clearpage

\begin{comment}
\addcontentsline{toc}{section}{Liite A}
\section{Esimerkki liitteestä\label{LiiteA}}
%% Liitteiden kaavat, taulukot ja kuvat numeroidaan omana kokonaisuutenaan
%%
%% Equations, tables and figures have their own numbering in Appendices
\renewcommand{\theequation}{A\arabic{equation}}
\setcounter{equation}{0}  
\renewcommand{\thefigure}{A\arabic{figure}}
\setcounter{figure}{0}
\renewcommand{\thetable}{A\arabic{table}}
\setcounter{table}{0}

Liitteet eivät ole opinnäytteen kannalta välttämättömiä ja 
opinnäytteen tekijän on 
kirjoittamaan ryhtyessään hyvä ajatella pärjäävänsä ilman liitteitä.
Kokemattomat kirjoittajat, jotka ovat huolissaan
tekstiosan pituudesta, paisuttavat turhan 
helposti liitteitä pitääkseen tekstiosan pituuden annetuissa rajoissa.
Tällä tavalla ei synny hyvää opinnäytettä.   

Liite on itsenäinen kokonaisuus, vaikka se täydentääkin tekstiosaa.
Liite ei siten ole pelkkä listaus, kuva tai taulukko, vaan 
liitteessä selitetään aina sisällön laatu ja tarkoitus. 

Liitteeseen voi laittaa esimerkiksi listauksia. Alla on 
listausesimerkki tämän liitteen luomisesta. 

%% Verbatim-ympäristö ei muotoile tai tavuta tekstiä. Fontti on monospace.
%% Verbatim-ympäristön sisällä annettuja komentoja ei LaTeX käsittele. 
%% Vasta \end{verbatim}-komennon jälkeen jatketaan käsittelyä.
\begin{verbatim}
    \clearpage
	\appendix
	\addcontentsline{toc}{section}{Liite A}
	\section*{Liite A}
	...
	\thispagestyle{empty}
	...
	tekstiä
	...
	\clearpage
\end{verbatim}

Kaavojen numerointi muodostaa liitteissä oman kokonaisuutensa:
\begin{eqnarray}
d \wedge A  &=& F, \label{liitekaava1}\\
d \wedge F  &=& 0. \label{liitekaava2}
\end{eqnarray}


\clearpage
\addcontentsline{toc}{section}{Liite B}
\section{Toinen esimerkki liitteestä\label{LiiteB}}

%% Liitteiden kaavat, taulukot ja kuvat numeroidaan omana kokonaisuutenaan
%%
%% Equations, tables and figures have their own numbering in Appendices
\renewcommand{\theequation}{B\arabic{equation}}
\setcounter{equation}{0}  
\renewcommand{\thefigure}{B\arabic{figure}}
\setcounter{figure}{0}
\renewcommand{\thetable}{B\arabic{table}}
\setcounter{table}{0}

Liitteissä voi myös olla kuvia, jotka
eivät sovi leipätekstin joukkoon:
%% Ympäristön figure parametrit htb pakottavat
%% kuvan tähän, eikä LaTeX yritä siirrellä niitä
%% hyväksi katsomaansa paikkaan. 
%% Ympäristöä center voi käyttää \centering-
%% komennon sijaan
%%
%% Example of a figure, note the use of htb parameters which force
%% the figure to be inserted here
\begin{figure}[htb]
\begin{center}
\includegraphics[height=8cm]{kuva2}
\end{center}
\caption{Kuvateksti, jossa on liitteen numerointi \label{liitekuva}}
\end{figure}
%%
Liitteiden taulukoiden numerointi on kuvien ja kaavojen kaltainen:
\begin{table}[htb]
\caption{Taulukon kuvateksti. \label{liitetaulukko}}
\begin{center}
\fbox{
\begin{tabular}{lp{0.5\linewidth}}
9.00--9.55  & Käytettävyystestauksen tiedotustilaisuus (osanottajat
ovat saaneet sähköpostitse valmistautumistehtävät, joten tiedotustilaisuus
voidaan pitää lyhyenä).\\
9.55--10.00 & Testausalueelle siirtyminen
\end{tabular}}
\end{center}
\end{table}
Kaavojen numerointi muodostaa liitteissä oman kokonaisuutensa:
\begin{eqnarray}
T_{ik} &=& -p g_{ik} + w u_i u_k + \tau_{ik},  \label{liitekaava3} \\
n_i    &=& n u_i + v_i.                        \label{liitekaava4}
\end{eqnarray}

\end{comment}

\end{document}